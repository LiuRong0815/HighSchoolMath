\section{2020 年 9 月 29 日答疑记录}

对于任意 $x$, $y\in\realnum$, 恒有 $(x-y)^2\geqslant 0$, 展开后移项得
\[x^2+y^2\geqslant 2xy\quad\text{即}\quad \frac{x^2+y^2}2\geqslant xy.\]
不等式中等号成立的条件是 $x=y$, 也记为: ``$=$'' 成立当且仅当 $x=y$. 把上面两个式子中的 $x$ 和 $y$ 分别换成 $\sqrt{x}$ 和 $\sqrt{y}$ (此时必须限制 $x$, $y\geqslant 0$), 可知
\[x+y\geqslant 2\sqrt{xy}\quad\text{即}\quad \frac{x+y}2\geqslant \sqrt{xy}.\]
因为 $\dfrac{x+y}2$ 与 $\sqrt{xy}$ 分别叫做非负实数 $x$ 与 $y$ 的\myindex{算术平均值}{算术平均值}与\myindex{几何平均值}{几何平均值}, 所以上面最后一个不等式也称为\myindex{均值不等式}{均值不等式}, 即
\[\text{均值不等式:}\ \frac{x+y}2\geqslant \sqrt{xy},\quad x,y\geqslant 0.\]
以上不等式均为常用不等式, 且都可以互相推出.

利用均值不等式 (或其等价形式) 可以很方便地求特殊形式的式子的最大 (小) 值. 例如, 若 $x>0$, 则 
\[x+\dfrac1x\geqslant 2\sqrt{x\cdot\dfrac1x}=2,\]
``$=$'' 成立当且仅当 $x=\dfrac1x$ 即 $x=1$. 一般的, 可得如下结论 (均设 $x$, $y\geqslant 0$):

(1) 若 $xy=L$ 为定值, 则 
\[x+y\geqslant 2\sqrt{xy}=2\sqrt{L},\] 所以 $x+y$ 的最小值为 $2\sqrt{L}$, ``$=$'' 成立当且仅当 $x=y=\sqrt{L}$;

(2) 若 $x+y=M$ 为定值, 则 
\[\sqrt{xy}\leqslant \frac{x+y}2=\frac{M}2\quad\text{即}\quad
  xy\leqslant \biggl(\frac{M}2\biggr)^2= \frac{M^2}4,\]
所以 $xy$ 的最大值为 $\dfrac{M^2}4$, ``$=$'' 成立当且仅当 $x=y=\dfrac{M}2$.

\begin{example}
  解答下列问题: 
  \begin{subproblem}
    \item 若 $x>0$, 求 $x+\dfrac4x$ 的取值范围;
    \item 若 $x$, $y\geqslant 0$, $xy=1$, 求 $x+2y$ 的取值范围;
    \item 若 $x$, $y\geqslant 0$, $2x+y=1$, 求 $xy$ 的取值范围.
  \end{subproblem}
\end{example}
\begin{solution}
  (1) 由均值不等式, 
  \[x+\dfrac4x\geqslant 2\sqrt{x\cdot\dfrac4x}=4,\] ``$=$'' 成立当且仅当 $x=\dfrac4{x}$ 即 $x=2$, 所以 $x+\dfrac4x\in[4,+\infty)$.
  
  (2) 由均值不等式, 
  \[\frac{2x+y}2\geqslant \sqrt{2x\cdot y}\quad\text{即}\quad
    2x+y\geqslant 2\sqrt2,\]
  ``$=$'' 成立当且仅当 $2x=y$, 结合 $xy=1$ 知 $x=\dfrac{\sqrt2}2$, $y=\sqrt2$. 所以 $xy\in[2\sqrt2,+\infty)$.
  
  (3) 由均值不等式, 
  \[\sqrt{2x\cdot y}\leqslant \frac{2x+y}2\quad\text{即}\quad
    xy\leqslant \frac18,\]
  ``$=$'' 成立当且仅当 $2x=y$, 结合 $2x+y=1$ 知 $x=\dfrac14$, $y=\dfrac12$. 又因为 $x$, $y\geqslant 0$, 所以 $xy\geqslant 0$, 即 $xy\in\biggl[0,\dfrac14\biggr]$.
\end{solution}

利用均值不等式求取值范围时, 一般把要求值的式子写在不等号前面, 等于固定值的式子写在不等号后面, 且有时需要适当的变形.
  
在应用均值不等式时, 必须限制所考虑的式子 (均值不等式中的 $x$ 与 $y$) 为非负实数, 例如, 对任意的 $x\in\realnum$, 不能断言 $x+\dfrac1x\geqslant 2$ (当 $x\leqslant 0$ 时, 不等号显然不成立); 且应考虑等号成立的条件, 例如, 对任意的 $x\geqslant 2$, 不能由 $x+\dfrac1x\geqslant 2$ 断言 $x+\dfrac1x\in[2,+\infty)$, 因为此时 ``$=$'' 成立的条件 ``$x=1$'' 与已知条件 $x\geqslant 2$ 冲突 (实际上, 此时  $x+\dfrac1x\in \biggl[\dfrac52,+\infty\biggr)$, 具体解法以后会学到).

有时, 要求取值范围的式子不能直接用均值不等式, 但可以通过变形化为能用均值不等式的形式. 

\begin{example}\label{exa:201016-1940}
  解答下列问题: 
  \begin{subproblem}
    \item 若 $x>-1$, 求 $x+\dfrac1{x+1}$ 的取值范围;
    \item 若 $0<x<4$, 求 $x(8-2x)$ 的取值范围;
    \item 若 $x<0$, 求 $x+\dfrac1{x}$ 的取值范围.
  \end{subproblem}
\end{example}
\begin{solution}
  (1) 由均值不等式, 
  \[x+\dfrac1{x+1}= (x+1)+\dfrac1{x+1}-1
    \geqslant 2\sqrt{x\cdot\dfrac1{x+1}}-1=1,\]
  ``$=$'' 成立当且仅当 $x+1=\dfrac1{x+1}$ 即 $x=0$, 所以 $x+\dfrac1{x+1}\in[1,+\infty)$.
  
  (2) 由均值不等式, 
  \[\sqrt{x(8-2x)}= \sqrt2\sqrt{x(4-x)}
    \leqslant \sqrt2\cdot\frac{x+(4-x)}2= 2\sqrt2,\]
  即 $x(8-2x)\leqslant 8$, ``$=$'' 成立当且仅当 $x=4-x$ 即 $x=2$. 又由 $0<x<4$ 和二次函数的性质可知, $x(8-2x)$ 的最小值在定义域端点处取到, 所以 $x(8-2x)\in(0,8]$. (此题也可以直接用二次函数的性质来解, 即考虑图象的开口方向、对称轴、定义域.)
  
  (3) 因为 $x<0$, 所以先考虑 $(-x)+\biggl(-\dfrac1{x}\biggr)$. 由均值不等式, 
  \[(-x)+\biggl(-\dfrac1{x}\biggr)
    \geqslant 2\sqrt{(-x)\cdot\biggl(-\dfrac1{x}\biggr)}=2,\]
  即 $x+\dfrac1x\leqslant -2$, ``$=$'' 成立当且仅当 $-x= -\dfrac1{x}$ 即 $x=-1$ (注意, 此时 $x<0$), 所以 $x+\dfrac1{x}\in (-\infty,-2]$.
\end{solution}

从解题过程可以看出, 此时解题的思路是通过适当改变常数项 (如例~\ref{exa:201016-1940} (1)) 或系数 (如例~\ref{exa:201016-1940} (2)(3)), 想办法{\bfseries 凑}出两个式子, 使它们的积 (如例~\ref{exa:201016-1940} (1)(3)) 或和 (如例~\ref{exa:201016-1940} (2)) 为定值.
