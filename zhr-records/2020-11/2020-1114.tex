\section{2020 年 11 月 14 日答疑记录}

\begin{example}\label{exa:201201-2000}
    求函数 $f(x)= \dfrac{x^2-x+4}x$ ($x>0$) 的最小值.
\end{example}
\begin{solution}
    方法一: 先拆项, 再用均值不等式, 即
    \[f(x)= x+\frac4x-1\geqslant 2\sqrt{x\cdot\frac4x}-1=3,\]
    ``$=$'' 成立当且仅当 $x=\dfrac4x$ 即 $x=2$ (因为 $x>0$). 所以 $f(x)$ 的最小值为 $3$.
    
    方法二: 直接对分子用均值不等式. 因为 $x^2+4\geqslant 2\sqrt{x^2\cdot 4}= 4x$, 所以
    \[f(x)= \dfrac{x^2+4-x}x\geqslant \frac{4x-x}x= 3,\]
    ``$=$'' 成立当且仅当 $x^2= 4$ 即 $x=2$. 所以 $f(x)$ 的最小值为 $3$.
    
    方法三: 考虑 ``对勾函数'' $g(x)=x+\dfrac4x$ 的图形 (参考 ``2020 年 10 月 31 日答疑记录'' 中第二个例子), 可知其在 $(0,2]$ 上单调递减, 在 $(2,+\infty)$ 上单调递增, 所以最小值为 $g(2)=4$. 又因为 
    \[f(x)=x+\frac4x-1=g(x)-1,\]
    所以 $f(x)$ 的最小值为 $f(2)=3$.
\end{solution}

例~\ref{exa:201201-2000} 中三个方法都是常见解法, 前两个方法都使用了均值不等式, 需注意该不等式及其中等号成立的前提条件 (参考 ``2020 年 9 月 29 日答疑记录''), 第三个方法需要对 ``对勾函数'' 的图形非常熟悉. 此外, 第三个方法是通用解法, 例如由图形可知, 函数 $h(x)=x+\dfrac4x$ 在 $[3,+\infty)$ 上的值域为 $[h(3),+\infty)= \biggl[\dfrac{13}3,+\infty\biggr)$, 而在 $[1,3]$ 上的值域为 $[h(2),h(1)]= [4,5]$. 对前例的第一种情形, 均值不等式的等号不成立, 而对后一种情形, 利用均值不等式只能求得最小值.

\begin{example}\label{exa:201201-2020}
    已知函数
    \[f(x)= \begin{cases}
        (4a-3)x+a+\dfrac12, & x<0,\\
        a^x, & x\geqslant 0.
    \end{cases}\]
    
    (1) 若函数的图形经过点 $\biggl(3,\dfrac18\biggr)$, 求 $a$ 的值.
    
    (2) 若对任意的 $x_1\neq x_2$, 都有 $\dfrac{f(x_1)-f(x_2)}{x_1-x_2}<0$ 成立, 求 $a$ 的取值范围.
\end{example}
\begin{solution}
    (1) 由题意, $f(3)= \dfrac18$ 即 $a^3=\dfrac18$, 所以 $a=\dfrac12$.
    
    (2) $\dfrac{f(x_1)-f(x_2)}{x_1-x_2}<0$ 表明 $x_1$, $x_2$ 的大小关系与 $f(x_1)$, $f(x_2)$ 的大小关系恰好反过来 (即若 $x_1<x_2$, 则 $f(x_1)>f(x_2)$), 也就是 $f(x)$ 单调递减. 再由 $f(x)$ 的解析式知 (参考 ``2020 年 10 月 31 日答疑记录'' 的第三个例子)
    \[\left\{\!\!\begin{array}{l}
        4a-3<0,\ 0<a<1,\\
        a+\dfrac12\geqslant a^0,
        \end{array}\right.\ \text{解得}\quad 0<a<\frac12,\]
    所以 $a\in\biggl(0,\dfrac12\biggr)$.
\end{solution}

例~\ref{exa:201201-2020} 中 $\dfrac{f(x_1)-f(x_2)}{x_1-x_2}<0$ 表明 $f(x)$ 单调递减, 类似的结论 (需理解并熟记) 还有
\[\begin{gathered}
    \dfrac{f(x_1)-f(x_2)}{x_1-x_2}>0
    \Leftrightarrow (f(x_1)-f(x_2))(x_1-x_2)>0
    \Leftrightarrow \text{$f(x)$ 单调递增,}\\
    \dfrac{f(x_1)-f(x_2)}{x_1-x_2}<0
    \Leftrightarrow (f(x_1)-f(x_2))(x_1-x_2)<0
    \Leftrightarrow \text{$f(x)$ 单调递减.}
\end{gathered}\]

\begin{example}\label{exa:201201-2030}
    计算: $\log_4 3\cdot \log_9 2- \log_{\frac12} 32$.
\end{example}
\begin{solution}
    由对数的运算法则 (参考 ``2020 年 11 月 8 日答疑记录'' 对数练习小节), 
    \[\begin{aligned}
        \log_4 3\cdot \log_9 2- \log_{\frac12} 32
        &= \frac{\ln 3}{\ln 4}\cdot \frac{\ln 2}{\ln 9}
            - \frac{\ln 32}{\ln\dfrac12}\\
        &= \frac{\ln 3}{2\ln 2}\cdot \frac{\ln 2}{2\ln 3}
            - \frac{5\ln 2}{-\ln 2}\\
        &= \frac14+5= \frac{21}4.
    \end{aligned}\]
\end{solution}

计算例~\ref{exa:201201-2020} 时, 其中的对数都化为以  $\mathrm{e}$ 为底, 也就是化为自然对数. 在计算时, 也可以都化为以 $10$ 为底, 即化为常用对数. 此外, 也可以由 $32= 2^5= \bigg(\dfrac12\biggr)^{-5}$ 知 $\log_{\frac12} 32= -5$.

\begin{example}
    已知函数 $f(x)=\dfrac{ax^2+b}x$, 且 $f(1)=2$, $f(2)=\dfrac52$.
    
    (1) 确定函数 $f(x)$ 的解析式, 并判断其奇偶性;
    
    (2) ({\bfseries 选学}) 用定义证明函数 $f(x)$ 在区间 $(-\infty,-1)$ 上单调递增;
    
    (3) 求满足 $f(1+2t^2)- f(3+t^2)<0$ 的实数 $t$ 的取值范围.
\end{example}
\begin{solution}
    (1) 由题意,
    \[\left\{\!\!\begin{array}{l}
        a+b=2,\\
        \dfrac{4a+b}2= \dfrac52,
       \end{array}\right.\ \text{解得}\quad
       \left\{\!\!\begin{array}{l}
        a=1,\\
        b=1,
       \end{array}\right.\]
    所以 $f(x)=\dfrac{x^2+1}x$. 因为
    \[f(-x)= \frac{(-x)^2+1}{-x}
        = -\frac{x^2+1}x= -f(x),\]
    且 $f(x)$ 的定义域为 $(-\infty,0)\cup (0,+\infty)$, 关于原点对称, 所以 $f(x)$ 为奇函数.
    
    (2) 任取 $x_1<x_2<-1$, 则
    \[\begin{aligned}
        f(x_1)-f(x_2)
        &= \frac{x_1^2+1}{x_1}- \frac{x_2^2+1}{x_2}
         = \frac{x_2(x_1^2+1)- x_1(x_2^2+1)}{x_1x_2}\\
        &= \frac{x_1^2 x_2- x_1 x_2^2+x_2- x_1}{x_1x_2}
         = \frac{x_1 x_2(x_1- x_2)- (x_1- x_2)}{x_1x_2}\\
        &= (x_1- x_2)\frac{x_1 x_2- 1}{x_1x_2}.
        \end{aligned}\]
    由 $x_1<x_2<-1$ 知,
    \[x_1- x_2<0,\quad x_1 x_2> 1\ \text{即}\ x_1 x_2- 1>0,\]
    所以 $f(x_1)-f(x_2)<0$, 说明函数 $f(x)$ 在区间 $(-\infty,-1)$ 上单调递增.
    
    (3) 不等式化为 $f(1+2t^2)< f(3+t^2)$. 因为 $1+2t^2>1$, $3+t^2>1$, 且
    \[f(x)= \dfrac{x^2+1}x= x+\dfrac1x\quad (\text{对勾函数})\]
    在 $(1,+\infty)$ 上单调递增, 所以前述不等式等价于 
    \[1+2t^2<3+t^2,\quad \text{解得}\ 
        t\in(-\infty,-\sqrt2)\cup(\sqrt2,+\infty).\]    
\end{solution}

\begin{example}\label{exa:201206-1300}
    已知函数 $f(x)$ 为奇函数, 且当 $x>0$ 时, $f(x)= x^2+\dfrac1x$, 求当 $x<0$ 时 $f(x)$ 的解析式.
\end{example}
\begin{solution}
    方法一: 若 $x<0$, 则 $-x>0$, 由题意, 此时 
    \[f(-x)= (-x)^2+\frac1{-x}= x^2-\frac1x.\]
    因为 $f(x)$ 为奇函数, 所以 $f(-x)= -f(x)$, 代入上式可得
    \[-f(x)= x^2-\frac1x,\quad\text{即}\quad f(x)=-x^2+\frac1x.\]
    此即为当 $x<0$ 时 $f(x)$ 的解析式.
    
    方法二 (将方法一的步骤压缩): 因为 $f(x)$ 为奇函数, 所以当 $x<0$ 时, 
    \[f(x)= -f(-x)= -\biggl(x^2-\frac1x\biggr)= -x^2+\frac1x.\]
\end{solution}

\begin{example}
    近年来大气污染防治工作得到各级部门的重视. 某企业每日生产总成本 $y$ (单位: 万元) 与日产量 $x$ (单位: 吨) 之间的函数关系式为
    \[y=2x^2+(15- 4k)x+120k+2.\]
    现为了配合环境卫生综合整治, 该企业引进了除尘设备, 每吨产品除尘费用为 $k$ 万元, 除尘后当日产量为 $1$~吨时, 生产总成本为 $253$~万元.
    
    (1) 求实数 $k$ 的值;\qquad
    (2) 若每吨产品出厂价为 $59$~万元, 并假设每天的产品均能卖出, 当除尘后日产量为多少时, 平均每吨产品的利润最大? 最大利润为多少?
\end{example}
\begin{solution}
    (1) 设除尘后的每日生产总成本为 $f(x)$ (单位: 万元), 则
    \[f(x)= y+kx= 2x^2+(15- 3k)x+120k+2.\]
    由题意, $f(1)= 253$, 所以
    \[2+(15- 3k)+120k+2= 253,\quad\text{解得}\ k=2.\]
    
    (2) 除尘后每天的收入为 $59x$ (单位: 万元), 所以利润为 $59x-f(x)$  (单位: 万元), 平均每吨产品的利润为
    \[\frac{59x-f(x)}x= 59- \frac{2x^2+(15- 6)x+240+2}x
        = 50- 2\biggl(x+\frac{121}x\biggr)\quad(\text{万元}).\]
    由均值不等式, $x+\dfrac{121}x\geqslant 22$, ``$=$'' 成立当且仅当 $x=11$, 则 
    \[\frac{59x-f(x)}x\leqslant 50- 2\cdot 22= 6,\]
    表明当日产量为 $11$~吨时, 平均每吨产品的利润最大, 且最大值为 $6$~万元.
\end{solution}