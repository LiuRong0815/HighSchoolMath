\section{2020 年 11 月 29 日答疑记录}

\begin{example}
    求下列函数的定义域、值域和单调区间:
    
    (1) $y=\dfrac{\sqrt{x}-1}{\sqrt{x}+1}$;\qquad
    (2) $y=\dfrac{1-3^x}{1+3^x}$.
\end{example}
\begin{solution}
    (1) $x$ 需满足 $\sqrt{x}+1\neq 0$, 即 $x\in[0,+\infty)$. 先将函数用 ``分离常数法'' 变形,
    \[y=\frac{(\sqrt{x}+1)-2}{\sqrt{x}+1}
        = 1-\frac{2}{\sqrt{x}+1}.\]
    因为 $\sqrt{x}\in[0,+\infty)$, 所以 
    \[\begin{gathered}
        \sqrt{x}+1\in[1,+\infty),\quad 
            \frac2{\sqrt{x}+1}\in(0,2],\\
        -\frac2{\sqrt{x}+1}\in[-2,0),\quad
        1-\frac2{\sqrt{x}+1}\in[-1,1),
    \end{gathered}\]
    即 $y\in[-1,1)$, 且按上述过程可知函数在 $[0,+\infty)$ 上单调递增.
    
    (2) $1+3^x\neq0$ 即 $x\in\realnum$. 由 
    \[y= \frac{-(1+3^x)+2}{1+3^x}= -1+\frac2{1+3^x}\]
    并仿 (1) 的过程知 $y\in(-1,1]$, 且在 $\realnum$ 上单调递减.
\end{solution}

分离常数法是对形如 $\dfrac{ax+b}{cx+d}$ 的式子变形的常用方法, 以下再举一些例子:
\[\begin{aligned}
    \frac{2x+1}{x+1}&= \frac{2(x+1)-1}{x+1}= 2-\frac1{x+1},\\
    \frac{x+2}{2x+1}&= \frac{\dfrac12(2x+1)+\dfrac32}{2x+1}= \frac12+\frac{\dfrac32}{2x+1},\\
    \frac{2x+1}{3x-1}&= \frac{\dfrac23(3x-1)+\dfrac53}{3x-1}= \frac23-\frac{\dfrac53}{x+1}.
\end{aligned}\]
除了可以用来求函数的值域, 该方法还可以用来画函数图形 (一般需要利用图形变换).

\begin{example}
    已知 $x+2y=1$, 求 $2^x+4^y$ 的最小值.
\end{example}
\begin{solution}
    因为 $x+2y=1$, 而由均值不等式,
    \[2^x+4^y\geqslant 2\sqrt{2^x\cdot 4^y}
        = 2\sqrt{2^{x+2y}}= 2\sqrt2,\]
    ``$=$'' 成立当且仅当 $2^x=4^y$ 即 $x=2y=\dfrac12$, 所以 $2^x+4^y$ 的最小值为 $2\sqrt2$.
\end{solution}

\begin{example}
    若指数函数 $f(x)=(a+1)^x$ 是 $\realnum$ 上的减函数, 求 $a$ 的取值范围.
\end{example}
\begin{solution}
    对指数函数 $y=a^x$ 而言, 其单调递减表明底数 $a\in(0,1)$, 此题中对应有 $a+1\in(0,1)$, 所以 $a\in(-1,0)$.
\end{solution}

\begin{example}
    设函数 $f(x)=\begin{cases}
        \biggl(\dfrac12\biggr)^x-1, & x<0,\\
        x^{\frac12}, & x\geqslant 0,
    \end{cases}$ 求不等式 $f(x)\geqslant 1$ 的解.
\end{example}
\begin{solution}
    (1) 若 $x<0$, 则不等式化为 
    \[\biggl(\frac12\biggr)^x-1\geqslant 1,\quad \text{即}\quad
        \biggl(\frac12\biggr)^x\geqslant 2= \biggl(\frac12\biggr)^{-1}.\]
    由函数 $y=\biggl(\dfrac12\biggr)^x$ 单调递减可知, $x\in(-\infty,-1]$.
    
    (2) 若 $x\geqslant 0$, 则不等式化为 $x^{\frac12}\geqslant 1$. 由函数 $y=x^{\frac12}$ 在 $[0,+\infty)$ 上单调递增可知, $x\in[1,+\infty)$.
    
    综上所述, $x\in(-\infty,-1]\cup [1,+\infty)$.
\end{solution}