\section{2020 年 12 月 16 日答疑记录}

    同角三角函数有两个常用基本关系式: 
  \[\sin^2 x+ \cos^2 x= 1,\quad 
    \tan x= \frac{\sin x}{\cos x}.\]
  后一个式子也可以认为是正切的定义. 前一个式子等价于勾股定理, 由该式可知
  \[\begin{gathered}
    (\sin x\pm \cos x)^2= 1\pm 2\sin x\cos x,\\
        \sin^2 x= 1-\cos^2 x= (1+\cos x)(1-\cos x).
  \end{gathered}\] 
  
\begin{example}
    已知 $\cos\alpha= -\dfrac35$, 且 $\tan\alpha>0$, 求 $\dfrac{\sin\alpha\cos^2\alpha}{1-\sin\alpha}$.
\end{example}
\begin{solution}
    方法一: 由题意, $\alpha$ 在第三象限, 所以 $\sin\alpha= -\dfrac45$, 
    \[\frac{\sin\alpha\cos^2\alpha}{1-\sin\alpha}
        = \frac{-\dfrac45\cdot\bigg(-\dfrac35\biggr)^2}{1- \biggl(-\dfrac45\biggr)}
        = -\frac{4}{25}.\]
    
    方法二: 也可以利用 $\cos^2\alpha= 1-\sin^2\alpha=(1+\sin\alpha)(1-\sin\alpha)$ 得,
    \[\frac{\sin\alpha\cos^2\alpha}{1-\sin\alpha}
        =\frac{\sin\alpha(1-\sin^2\alpha)}{1-\sin\alpha}
        = \sin\alpha(1+\sin\alpha)
        = -\frac{4}{25}.\]
\end{solution}

\begin{example}
    化简 $\dfrac{\cos\theta}{1+\cos\theta}- \dfrac{\cos\theta}{1-\cos\theta}$.
\end{example}
\begin{solution}
    通分后合并可知,
    \[\begin{aligned}
        \frac{\cos\theta}{1+\cos\theta}- \frac{\cos\theta}{1-\cos\theta}
        &= \frac{\cos\theta(1- \cos\theta- (1+ \cos\theta))}{(1- \cos\theta)(1+ \cos\theta)}\\
        &= \frac{\cos\theta(-2\cos\theta)}{1- \cos^2\theta}
         = -\frac{2\cos^2\theta}{\sin^2\theta}\\
        &= - \frac2{\tan^2\theta}.
    \end{aligned}\]
    其中最后一个等号及其后的式子可以不写.
\end{solution}

  凡正余弦的二次式, 均可以化成正切函数来表示, 例如:
  \[\sin x\cos x+ \cos^2 x
    = \frac{\sin x\cos x+ \cos^2 x}{\sin^2 x+ \cos^2 x}
    = \frac{\tan x+ 1}{\tan^2 x+ 1}.\]
  与此类似的还有
  \[\frac{\sin x+\cos x}{\sin x-\cos x}
    = \frac{\tan x+1}{\tan x-1}.\]
  这两种变形方法常用来解决正余弦值和正切值的转化问题.
  
\begin{example}
    已知 $\dfrac{\sin\theta+ \cos\theta}{\sin\theta- 2\cos\theta}= \dfrac12$, 求 $\tan\theta$.
\end{example}
\begin{solution}
    方法一: 原式分子、分母同除以 $\cos\theta$, 得 $\dfrac{\tan\theta+ 1}{\tan\theta- 2}= \dfrac12$, 所以 $\tan\theta= -4$.
    
    方法二: 将原式化为整式, 
    \[2(\sin\theta+ \cos\theta)=\sin\theta- 2\cos\theta,\]
    所以 $\sin\theta= -4\cos\theta$, 即 $\tan\theta= -4$.
\end{solution}

\begin{example}
    设角~$\alpha$ 的终边过点~$P(3,4)$, 求 $\dfrac{\sin\alpha+ 2\cos\alpha}{\sin\alpha- \cos\alpha}$.
\end{example}
\begin{solution}
    方法一: 由题意, $\tan\alpha= \dfrac43$, 所以
    \[\frac{\sin\alpha+ 2\cos\alpha}{\sin\alpha- \cos\alpha}
        = \frac{\tan\alpha+ 2}{\tan\alpha- 1}
        = 10.\]
    
    方法二: 由题意, $\dfrac{\sin\alpha}{\cos\alpha}= \dfrac43$, 即 $\sin\alpha= \dfrac43\cos\alpha$, 所以
    \[\frac{\sin\alpha+ 2\cos\alpha}{\sin\alpha- \cos\alpha}
        = \frac{\dfrac43\cos\alpha+ 2\cos\alpha}{\dfrac43\cos\alpha- \cos\alpha}
        = 10.\]
\end{solution}