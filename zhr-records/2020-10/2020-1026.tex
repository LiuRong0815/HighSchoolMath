\section{2020 年 10 月 26 日答疑记录}


\begin{example}\label{201109-2110}
    已知 $A=\{x\mid y=\sqrt{9-x^2}\}$, $B=\{y\mid y=x^2-2x+3\}$, 求 $A\cap B$.
\end{example}
\begin{solution}
    集合~$A$ 表示 $y=\sqrt{9-x^2}$ 中 $x$ 的范围, 即该函数的定义域 $[-3,3]$, 集合~$B$ 表示 $y=x^2-2x+3$ 中 $y$ 的范围, 即该函数的值域 $[2,+\infty)$, 所以 $A\cap B= [2,3]$.
\end{solution}
\begin{remark}
    解集合问题时, 一定要注意集合的意义. 例~\ref{201109-2110} 中的集合~$A$ 表示的是定义域 (描述的是 $x$), 而集合~$B$ 表示的是值域 (描述的是 $y$), 所以应分别按定义域和值域的求法确定这两个集合.
\end{remark}

\begin{example}\label{201109-2120}
    已知函数 $f(x)= 4x^2-kx-8$ 在 $[5,20]$ 上单调变化, 求实数 $k$ 的取值范围.
\end{example}
\begin{solution}
    $f(x)$ 为二次函数, 单调性由开口方向、对称轴和定义域共同决定. 由题意, $f(x)$ 的对称轴 $x=\dfrac{k}8$ 不在区间 $[5,20]$ 内, 所以
    \[\frac{k}8\leqslant 5\ \text{或} \frac{k}8\geqslant 20,\quad
        \text{解得}\ k\leqslant 40\ \text{或} k\geqslant 160,\]
    即 $k\in(-\infty,40]\cup [160,+\infty)$.
\end{solution}
\begin{remark}
    例~\ref{201109-2120} 中的二次函数在区间 $[5,20]$ 上若单调递增, 则 $\dfrac{k}8\leqslant 5$; 若单调递减, 则 $\dfrac{k}8\geqslant 20$ (为什么?).
\end{remark}


\begin{example}
    ``$x^2>1$'' 是 ``$x>1$'' 的 \underline{\qquad} 条件.
\end{example}
\begin{solution}
    由 $x^2>1$ 知 $x^2-1>0$, 解得 $x<-1$ 或 $x>1$, 所以 ``$x^2>1$'' 是 ``$x>1$'' 的必要不充分条件.
\end{solution}
\begin{remark}
    由 $x^2>1$ 不能得出 $x>\pm1$ (与方程不同). 解此不等式时, 建议用普通二次不等式的解法, 即先整理为和 $0$ 比较, 再因式分解.
\end{remark}

