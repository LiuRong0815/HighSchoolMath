\section{2020 年 10 月 6 日答疑记录}

\subsection{集合的子集的个数}

\begin{example}\label{exa:201021-1935}
  已知集合 $A=\{0,1,2\}$, 求 $A$ 的非空子集的个数.
\end{example}
\begin{solution}
  $A$ 的子集是从 $A$ 中取一些元素构成的集合, 可以不取 (对应空集 $\varnothing$), 也可以全取 (对应 $A$ 本身). 因为 $0$, $1$, $2$ 都有 ``取'' 或 ``不取'' 两种选择, 所以 $A$ 的子集共有 $2\times 2\times 2= 2^3$ 种可能. 非空子集不包括空集 $\varnothing$, 所以共有 $2^3-1=7$ 种可能.
\end{solution}

例~\ref{exa:201021-1935} 中的结论可以一般化: 

\begin{theorem}
  若 $A=\{a_1,a_2,\cdots,a_n\}$ 是含 $n$ 个元素的集合, 则 $A$ 有 $2^n$ 个子集, $(2^n-1)$ 个非空子集 (不含空集 $\varnothing$), $(2^n-1)$ 个真子集 (不含 $A$ 本身), $(2^n-2)$ 个非空真子集 (不含 $A$ 本身和空集 $\varnothing$).
\end{theorem}

其他类型的子集个数的问题, 可以考虑先转化为能利用该结论来解答的形式.

\begin{example}\label{exa:201021-1945}
  已知集合 $M$ 满足 $\{1,2\}\subseteq M\subseteq \{1,2,3,4,5\}$, 求这样的 $M$ 的个数.
\end{example}
\begin{solution}
  由题意, $M$ 中必含 $1$, $2$, 所以只需考虑 $M$ 中是否含有 $3$, $4$, $5$. 易知此时 $M$ 有 $2^3=8$ 种可能.
\end{solution}

在例~\ref{exa:201021-1945} 中若已知 $\{1,2\}\subset M\subseteq \{1,2,3,4,5\}$, 或 $\{1,2\}\subseteq M\subset \{1,2,3,4,5\}$, 则所求 $M$ 的个数均为 $2^3-1=7$ 个; 若已知 $\{1,2\}\subset M\subset \{1,2,3,4,5\}$, 则所求 $M$ 的个数均为 $2^3-2=6$ 个.

\subsection{集合的包含关系}

\begin{example}\label{exa:201021-1955}
  已知集合 $A=\{-1,1\}$, $B=\{x\mid mx=1\}$, 且 $B\subseteq A$, 求实数 $m$ 的值.
\end{example}
\begin{solution}
  集合 $B$ 表示关于 $x$ 的方程 $mx=1$ 的解集, 而该方程至多只有一个解 (为什么?). 因为 $B\subseteq A$, 所以 $B=\varnothing$, $\{-1\}$ 或 $\{1\}$, 对应的, $m=0$, $-1$ 或 $1$.
\end{solution}

在解集合的包含关系的问题 (如例~\ref{exa:201021-1955}) 时, 除了要考虑集合本身的含义 (如该例中 $B$ 表示一元一次方程的解集), 务必注意子集 (该例中的 $B$) 可能是空集.

\begin{example}\label{exa:201021-2005}
  已知集合 $A=\{1,3,m\}$, $B=\{3,m^2\}$, 若 $B\subsetneqq A$, 求实数 $m$ 的值.
\end{example}
\begin{solution}
  由题意, $B$ 中的 $m^2$ 是 $A$ 中的元素, 且不为 $3$.
  
  (1) 若 $m^2=1$, 则 $m=\pm1$. 当 $m=1$ 时, $A=\{1,3,1\}$, 不符合集合的定义 (不满足互异性); 当 $m=-1$ 时, $A=\{1,3,-1\}$, $B=\{3,1\}$, 符合题意.
  
  (2) 若 $m^2=m$, 则 $m=1$ 或 $0$, 且 $m=1$ 已符合题意. 当 $m=0$ 时, $A=\{1,3,0\}$, $B=\{3,0\}$, 也符合题意.
  
  综上所述, $m=0$ 或 $1$.
\end{solution}

解含参数 (如例~\ref{exa:201021-2005} 中的 $m$) 的集合包含问题, 需要根据集合的相关定义讨论, 算出参数的值之后, \myemph{必须回代检验}, 即检查是否满足题中的包含关系, 以及是否符合集合的定义 (互异性及无序性).

\begin{example}
  设 $A=\{x\mid x^2-5x+6=0\}$, $B=\{x\mid ax-1=0\}$. 若 $B\subsetneqq A$, 求实数 $a$ 的取值集合.
\end{example}
\begin{solution}
  显然 $A=\{2,3\}$, $B$ 表示关于 $x$ 的方程 $ax-1=0$ 的解集, 该方程至多有一个解. 由 $B\subsetneqq A$ 知, $B=\varnothing$, $\{2\}$ 或 $\{3\}$.
  
  若 $B=\varnothing$, 则 $a=0$; 若 $B=\{2\}$, 则 $a=\dfrac12$; 若 $B=\{3\}$, 则 $a=\dfrac13$. 所以 $a$ 的取值集合为 $\biggl\{0, \dfrac13,\dfrac12\biggr\}$.
\end{solution}


\endinput