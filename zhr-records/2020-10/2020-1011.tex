\section{2020 年 10 月 11 日答疑记录}

此次答疑为月考试卷部分习题解析, 个别题的叙述略有改变. 为方便起见, 略去了一些图示, 可自行补画草图.

\begin{example}
  若 $a>b$, 则 (\qquad).
  \begin{fourcolpro}
  A. $\sqrt[3]{a}>\sqrt[3]{b}$ & B. $\sqrt{a}>\sqrt{b}$
  & C. $a^3>b^2$ & D. $a^2>b^3$
  \end{fourcolpro}
\end{example}
\begin{solution}
  (1) 由 $f(x)=\sqrt[3]{x}$ ($x\in\realnum$) 的函数图象 (可通过描点作图得到) 知, $f(x)$ 为增函数, 所以 $a>b\Leftrightarrow \sqrt[3]{a}>\sqrt[3]{b}$.
  
  (2) 函数 $g(x)=\sqrt{x}$ 也为增函数, 但其定义域为 $[0,+\infty)$ (非负实数), 所以当 $a>b$ 时, $a$, $b$ 不一定在定义域内, 不能得出 $\sqrt{a}>\sqrt{b}$.
  
  (3) 取 $a=-1$, $b=-2$ 可知, 此时 $a>b$, 但 $a^3=-1<4=b^2$.
  
  (4) 取 $a=4$, $b=3$ 可知, 此时 $a>b$, 但 $a^2=16<27=b^3$. (也可以这样理解: 对于比较大的正数, 其三次方的增长速度比其二次方的增长速度要快得多, 所以较小正数的三次方也能大于较大正数的二次方.)
\end{solution}

\begin{example}\label{exa:201026-1913}
  设 $a\in\realnum$ 且 $a\neq 0$, 则四个数 $a^3+1$, $a^4-a^2+2$, $a+\dfrac1a$, $a^2+\dfrac1{a^2}$ 中, 恒大于 $1$ 的数有哪些?
\end{example}
\begin{solution}
  由 $f(x)=x^3+1$ 的图象 (可由 $g(x)=x^3$ 的图象向上平移一个单位长度得到) 可知, 其值域为 $\realnum$, 所以 $a^3+1=f(a)\in\realnum$, 不恒大于 $1$.
  
  因为 
  \[a^4-a^2+2=\biggl(a^2-\frac14\biggr)^2+\frac74\geqslant \frac74,\]
  所以 $a^4-a^2+2$ 恒大于 $1$.
  
  由均值不等式的等价形式 $x+y\geqslant 2\sqrt{xy}$ ($x$, $y\geqslant 0$) 知, 当 $a>0$ 时, $a+\dfrac1a\geqslant 2$; 当 $a<0$ 时, $a+\dfrac1a\leqslant -2$, 所以 $a+\dfrac1a$ 不恒大于 $1$.
  
  同理可知, $a^2+\dfrac1{a^2}\geqslant 2$ (注意, $a^2>0$).
\end{solution}

恒成立问题基本都应该转化为值域问题来解决, 对于不同的题型, 有相应的快速解法: 二次函数应先考虑对称轴和开口方向, 简单的题也可以直接配方 (如例~\ref{exa:201026-1913} 中的 $a^4-a^2+2$ 可视为 $a^2$ 的二次函数); 形如 $x+\dfrac{k}x$ (常数 $k>0$) 的式子, 优先考虑均值不等式 (需注意检验等号成立的条件, 该例中暂时略掉了).

\begin{example}
  某房地产公司计划出租 $70$~套相同的公寓,当每套公寓月租金定为 $3000$~元时, 这 $70$~套公寓能全租出去; 当月租金每增加 $50$~元时 (设月租金为 $50$~元的整数倍), 就会多一套公寓不能出租. 设已出租的每套公寓当月需要花费 $100$~元的日常维修等费用 (设未出租的公寓不需要花这笔费用). 要使公司获得最大利润, 每套公寓月租金应定为多少元?
\end{example}
\begin{solution}
  由题意可设租金应在 $3000$~元的基础上增加 $x$~个 $50$~元, 此时租金为 $(3000+50x)$~元, 出租 $(70-x)$~套公寓. 再记出租公寓的月收入为 $y$~元, 则
  \[y=(3000+50x-100)(70-x)= 50(x+58)(70-x).\]
  上式为二次函数的两根式且图象开口向上, 对称轴为 $x=\dfrac{-58+70}2=6$, 所以每套公寓月租金应定为 $3000+50\cdot 6= 3300$~元.
\end{solution}
\begin{answer}
  每套公寓月租金应定为 $3300$~元.
\end{answer}

\begin{example}\label{exa:201026-1933}
  条件 ``$a=-1$'' 是 ``函数 $y=ax^2+2x-1$ 的图象与 $x$~轴只有一个交点'' 的什么条件?
\end{example}
\begin{solution}
  对函数 $y=ax^2+2x-1$, 
  
  若 $a=0$, 则其为一次函数, 图象与 $x$~轴只有一个交点; 
  
  若 $a\neq 0$, 则其为二次函数, 只需 $\Delta= 2^2-4a\cdot(-1)=0$, 解得 $a=-1$. 反之, 容易验证当 $a=0$ 或 $-1$ 时, 函数 $y=ax^2+2x-1$ 的图象与 $x$~轴只有一个交点. 
  
  所以条件 ``$a=-1$'' 是 ``函数 $y=ax^2+2x-1$ 的图象与 $x$~轴只有一个交点'' 的充分不必要条件.
\end{solution}

判断充要性 (充分条件、必要条件) 的问题, 仍然是先考虑参数 (例~\ref{exa:201026-1933} 中的 $a$) 的取值范围. 解该例还需注意, 形如 $f(x)=Ax^2+Bx+C$ 的函数, \myemph{必须}先考虑 $A$ 是否为零, 即要先判断该函数到底是一次函数还是二次函数.

\begin{example}
  使不等式 $1+\dfrac1x>0$ 成立的一个充分不必要条件是 (\qquad).
  \begin{twocolpro}
  A. $x>2$ & B. $x>0$ \\
  C. $x<-1$ 或 $x>1$ & D. $-1<x<0$
  \end{twocolpro}
\end{example}
\begin{solution}
  不等式 $1+\dfrac1x>0$ 化为 
  \[\dfrac{1+x}x>0\quad\text{即}\quad (1+x)x>0,\]
  也就是 $x<-1$ 或 $x>0$. 该条件的充分不必要条件中 $x$ 的范围应更小一些, 结合数轴表示范围, 选项中 A, B, C 符合题意.
\end{solution}

\begin{example}
  下列命题中是假命题的有 (\qquad).
  
  A. $|x|^2+|x|-2=0$ 有四个实数解
  
  B. 若函数 $y=x^2-2ax+1$ 当 $2<x<3$ 时为增函数, 则 $a>2$
  
  C. 若 $x^2-3x+2\neq 0$, 则 $x\neq 2$
  
  D. 若 $x\in\realnum$, 则函数 $y=\sqrt{x^2+4}+\dfrac1{\sqrt{x^2+4}}$ 的最小值为 $2$   
\end{example}
\begin{solution}
  对 A, 方程为 $(|x|+2)(|x|-1)=0$, 则 $|x|=-2$ (舍) 或 $1$, 所以 $x=\pm1$, 原方程只有两个实数解.
  
  对 B, 可知函数图象的对称轴 $x=a$ 在定义域 $(2,3)$ 左侧, 故 $a\leqslant 2$.
  
  对 C, 不等式化为 $(x-1)(x-2)\neq 0$, 所以 $x\neq 1$ \myemph{且} $x\neq 2$.
  
  对 D, 由均值不等式, $y\geqslant 2$. 但等号成立的条件是 
  \[\sqrt{x^2+1}=\dfrac1{\sqrt{x^2+4}}\quad\text{即}\quad x^2+4=1\ (\text{无解}),\]
  说明前述等号无法成立, 即只能判断 $y>2$. (可以证明, 此时 $y\geqslant \dfrac52$, 即最小值为 $\dfrac52$, 等号成立当且仅当 $x=0$.)
\end{solution}

\begin{example}
  已知某二次函数满足: 当 $x=2$ 时, $y=-1$; 当 $x=-1$ 时, $y=-1$, 且最大值为 $8$, 求此二次函数的解析式.
\end{example}
\begin{solution}
  方法一(通用解法): 设此二次函数的解析式为 $f(x)=Ax^2+Bx+C$. 由题意, $f(2)=f(-1)=-1$, $A<0$ 且顶点纵坐标 $\dfrac{4AC-B^2}{4A}=8$. 解前述方程可得答案.
  
  方法二(特殊解法): 由题意, 点 $(2,-1)$, $(-1,-1)$ 均在二次函数图象上, 而这两个点纵坐标相同, 所以图象的对称轴为 $x=\dfrac{2+(-1)}2=\dfrac12$. 再由二次函数最大值为 $8$, 可直接设顶点式 $f(x)=A\biggl(x-\dfrac12\biggr)^2+8$, 再把点 $(2,-1)$ 代入, 可解得 $A$ 的值.
\end{solution}


\begin{example}\label{exa:201026-2120}
  已知二次函数 $y=-x^2+2ax+1-a$ 当 $0\leqslant x\leqslant 1$ 时有最大值 $2$, 求 $a$ 的值.
\end{example}
\begin{solution}
  函数 $y=-x^2+2ax+1-a$ 图象的对称轴为 $x=a$, 且开口向下.
  
  (1) 当 $a\leqslant 0$ 时, $y_{\max}= y(0)$ 即 $2=1-a$, 解得 $a=-1$.
  
  (2) 当 $0<a\leqslant 1$ 时, $y_{\max}= y(a)$ 即 $2=a^2+1-a$, 解得 $a=\dfrac{-1\pm\sqrt5}2$. 因为 $\sqrt5\approx 2.236$, 所以 $a\approx 0.618$ 或 $-1.618$. 由 $0<a\leqslant \dfrac12$ 知, $a=\dfrac{\sqrt5-1}2$.
  
  (3) 当 $a>1$ 时, $y_{\max}= y(1)$ 即 $2=a$, 所以 $a=2$.
  
  综上所述, $a=-1$, $\dfrac{\sqrt5-1}2$ 或 $2$.
\end{solution}
\begin{remark}
  (1) 一般二次函数的值域讨论, 有四种情形. 但例~\ref{exa:201026-2120} 中只考虑最大值, 且图象开口向下, 所以可以精简为三种情形 (轴在定义域内的两种情形并为一种).
  
  (2) 建议记住三个常用的算术平方根的近似值: $\sqrt2\approx 1.414$, $\sqrt3\approx 1.732$, $\sqrt5\approx 2.236$. 如果未记住 $\sqrt5$ 的近似值, 也可估计 $\sqrt5\in(2,3)$, 然后对所的式子进一步估值 (如例~\ref{exa:201026-2120} (2) 中的式子).
\end{remark}

\begin{example}\label{exa:201026-2130}
  设二次函数 $y=\dfrac12 x^2-x-\dfrac12$, 若当 $-1\leqslant x\leqslant m$ 时, $-1\leqslant y\leqslant m$, 求 $m$ 的值.
\end{example}
\begin{solution}
  函数 $y=\dfrac12 x^2-x-\dfrac12$ 图象开口向上, 对称轴为 $x=1$. 由题意, $m\geqslant -1$.
  
  方法一: (1) 当 $-1\leqslant m\leqslant 1$ 时, 
  \[\left\{\!\!\begin{array}{l}
    y_{\min}= y(m),\\
    y_{\max}= y(-1),
    \end{array}\right.\ \text{即}
    \left\{\!\!\begin{array}{l}
    -1= \dfrac12 m^2-m-\dfrac12,\\
    m= 1,
    \end{array}\right.\]
  解得 $m=1$.
  
  (2) 当 $1< m\leqslant 3$ 时, 
  \[\left\{\!\!\begin{array}{l}
    y_{\min}= y(1),\\
    y_{\max}= y(-1),
    \end{array}\right.\ \text{即}
    \left\{\!\!\begin{array}{l}
    -1= -1,\\
    m= 1,
    \end{array}\right.\]
  解得 $m=1$ (舍).
  
  (3) 当 $m>3$ 时, 
  \[\left\{\!\!\begin{array}{l}
    y_{\min}= y(1),\\
    y_{\max}= y(m),
    \end{array}\right.\ \text{即}
    \left\{\!\!\begin{array}{l}
    -1= -1,\\
    m= \dfrac12 m^2-m-\dfrac12,
    \end{array}\right.\]
  解得 $m=2\pm\sqrt3$. 由 $m>3$ 知 $m=2+\sqrt3$.
  
  综上所述, $m=1$ 或 $\dfrac{2+\sqrt3}2$.
  
  方法二: 因为 
  \[y=\frac12 x^2-x-\frac12= \frac12(x-1)^2-1,\]
  所以由 $-1\leqslant y\leqslant m$ 知 $m\geqslant 1$. 可以只讨论 $1\leqslant m\leqslant 3$ 和 $m>3$ 两种情形. 具体计算同方法一.
\end{solution}

例~\ref{exa:201026-2130} 仍为常见的求值域问题, 只需注意将定义域与对应的值域和题中的范围对比. 
