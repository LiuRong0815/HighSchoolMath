\section{2020 年 10 月 19 日答疑记录}

本次答疑主要讲解常见的定义域的求法. 如果没有特殊说明, 函数的定义域是使函数的表达式有意义的自变量的取值范围. 现阶段常见的表达式对其中的变量限制如下:
\begin{align*}
    \text{二次根式 $\sqrt{x}$}&\colon \text{被开方数非负, 即 $x\geqslant 0$;}\\
    \text{分式 $\dfrac1{x}$}&\colon \text{分母不为零, 即 $x\neq 0$;}\\
    \text{零次式 $x^0$}&\colon \text{底数不为零, 即 $x\neq 0$,}
\end{align*}
其中的零次式 $x^0$ 定义为 $\dfrac{x}x$, 所以需要限制 $x\neq 0$. 这些限制可以组合使用, 如对 $\dfrac1{\sqrt{x}}$, 应限制
\[\begin{cases}
    \sqrt{x}\neq 0 & (\text{分母不为零}),\\
    x\geqslant 0   & (\text{被开方数非负}),
    \end{cases}\quad\text{解得}\quad x>0.\]
上述结论可以直接使用. 类似地, 对 $\sqrt{\dfrac1x}$, 同样应限制 $x>0$.

\begin{example}\label{exa:201101-1415}
    求下列函数的定义域:
    \begin{twocolpro}
        (1) $f(x)=\dfrac{\sqrt{x-3}}{|x+1|-5}$;
        &(2) $f(x)=\dfrac{(x+1)^0}{|x|-x}$;\\
        (3) $f(x)=\dfrac1{1+\dfrac2x}$. &
    \end{twocolpro}
\end{example}
\begin{solution}
    (1) 由题意,
    \[\left\{\!\!\begin{array}{l}
        x-3\geqslant 0,\\
        |x+1|-5\neq0,
        \end{array}\right.\ \text{解得}\quad
        x\geqslant 3\ \text{且}\ x\neq4,\]
    所求定义域为 $[3,4)\cup(4,+\infty)$.
    
    (2) 由题意,
    \[\left\{\!\!\begin{array}{l}
        x+1\neq 0,\\
        |x|-x\neq0,
        \end{array}\right.\ \text{解得}\quad
        x<0\ \text{且}\ x\neq -1,\]
    所求定义域为 $(-\infty,-1)\cup(-1,0)$.
    
    (3) 由题意,
    \[\left\{\!\!\begin{array}{l}
        1+\dfrac2x\neq 0,\\
        x\neq0,
        \end{array}\right.\ \text{解得}\quad
        x\neq 0\ \text{且}\ x\neq -2,\]
    所求定义域为 $(-\infty,-2)\cup (-2,0)\cup (0,+\infty)$.
\end{solution}

例~\ref{exa:201101-1415} 的 (2) 中, 由 $|x|-x\neq 0$ 可得 $x<0$, 因为此时 $|x|\neq x$ 只在 $x<0$ 时才成立.

对于其他问题, 需要根据题意对自变量增加更多的限制, 例如三角形三边中任意两边之和大于第三边; 在实际问题中, 人数只能为非负整数, 等等.

在求抽象函数 (无具体表达式) 的定义域时, 需要注意函数的作用范围和定义域的区别. 函数是一个数集到另一个数集的对应关系, 前一个数集就是函数的作用范围, 不随表达式的变化而变化. 

例如, 对定义在 $(0,2)$ 上的函数 $f(x)$, 其定义域和作用范围均为 $(0,2)$ (两者相同). 再考虑 $f(2x)$, 此时函数 $f$ 的作用范围仍为 $(0,2)$ 且 $f$ 作用在 $2x$ 上, 所以 $2x$ 应在作用范围内, 即满足 $2x\in(0,2)$, 解得 $x\in(0,1)$, 表明 $f(2x)$ 的定义域 (即 $x$ 的取值范围) 为 $(0,1)$ (与作用范围不同). 注意, 这里 $f(x)$ 和 $f(2x)$ 中的 $x$ 含义不同, 其取值范围分别是对应函数的定义域.

再如, 对定义在 $(0,2)$ 上的函数 $g(x+1)$, 其定义域 (即 $x$ 的取值范围) 是 $(0,2)$, 但函数 $g$ 作用在 $x+1$ 上, 由 $x+1\in(1,3)$ 知其作用范围为 $(1,3)$ (与定义域不同). 再考虑 $g(x)$, 此时 $g$ 作用在 $x$ 上, 所以 $x$ 应在作用范围内, 即满足 $x\in(1,3)$, 表明 $g(x)$ 的定义域为 $(1,3)$ (与作用范围相同).

以上两个例子说明, 函数 $f(x)$ 的定义域与作用范围相同, 而函数 $f(ax+b)$ 的定义域与作用范围一般不同, 但是两个 $f$ 的作用范围相同, 即 $f(x)$ 中的 $x$ 与 $f(ax+b)$ 中的 $ax+b$ 都在同一取值范围内. 解题时\myemph{只需关注函数的作用范围不变即可} (即先求出该范围), 并注意定义域是当前表达式中 $x$ 的取值范围, 解题过程可适当简化.

\begin{example}
    已知函数 $f(x)$ 的定义域为 $[0,1)$, 求 $f(2x)$ 的定义域和 $f(x+3)$ 的定义域.
\end{example}
\begin{solution}
    对 $f(2x)$, 由已知 $2x\in[0,1)$, 所以 $x\in\biggl[0,\dfrac12\biggr)$, 即 $f(x)$ 的定义域为 $\biggl[0,\dfrac12\biggr)$.
    
    而对 $f(x+3)$, 有 $x+3\in[0,1)$, 所以 $x\in[-3,-2)$, 即 $f(x+3)$ 的定义域为 $[-3,-2)$.
\end{solution}

\begin{example}
    已知函数 $f(2x)$ 的定义域为 $[0,1)$, 求 $f(x)$ 的定义域和 $f(x+3)$ 的定义域.
\end{example}
\begin{solution}
    对 $f(2x)$, 由已知 $x\in[0,1)$, 所以 $2x\in[0,2)$, 即 $f(x)$ 的定义域为 $[0,2)$.
    
    而对 $f(x+3)$, 有 $x+3\in[0,2)$, 所以 $x\in[-3,-1)$, 即 $f(x+3)$ 的定义域为 $[-3,-1)$.
\end{solution}

\begin{example}
    已知函数 $f(x+3)$ 的定义域为 $[0,1)$, 求 $f(x)$ 的定义域和 $f(2x)$ 的定义域.
\end{example}
\begin{solution}
    对 $f(x+3)$, 由已知 $x\in[0,1)$, 所以 $x+3\in[3,4)$, 即 $f(x)$ 的定义域为 $[3,4)$.
    
    而对 $f(2x)$, 有 $2x\in[3,4)$, 所以 $x\in\biggl[\dfrac32,2\biggr)$, 即 $f(x+3)$ 的定义域为 $\biggl[\dfrac32,2\biggr)$.
\end{solution}

