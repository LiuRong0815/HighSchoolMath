\section{2020 年 10 月 4 日答疑记录}

\subsection{二次函数的图象}

\begin{example}
  已知二次函数 $y=ax^2+bx+10$ 当 $x=3$ 时的函数值与当 $x=2017$ 时的函数值相等, 求当 $x=2020$ 时的函数值.
\end{example}
\begin{solution}
  由题意, 函数图象的对称轴 $x=-\dfrac{b}{2a}=\dfrac{3+2017}2=2010$.
  
  方法一: 上式化为 $b=-2020a$, 所以当 $x=2020$ 时, 
  \begin{align*}
    y&= a\cdot 2020^2+b\cdot 2020+10\\
     &= a\cdot 2020^2-2020a\cdot 2020+10
      = 10.\end{align*}
  
  方法二: 画草图易知, 点 $x=2020$ 关于对称轴 $x=1010$ 的对称点为 $x=0$, 此时函数值为 二次函数的常数项, 即 $0$.
\end{solution}

\begin{example}
  已知抛物线 $y=(x-c)(x-d)-4$ 与 $x$~轴的交点为 $(6,0)$ 和 $(1,0)$, 求 $c$, $d$ 的值.
\end{example}
\begin{solution}
  方法一: 由已知可直接写出抛物线的两根式为 
  \[y=(x-6)(x-1),\]
  所以
  \[(x-c)(x-d)-4=(x-6)(x-1)\]
  即
  \[(x-c)(x-d)= x^2-7x+10=(x-2)(x-5),\]
  所以 $c=2$, $d=5$ 或 $c=5$, $d=2$.
  
  方法二: 将点 $(6,0)$ 和 $(1,0)$ 的坐标代入抛物线表达式,
  \[\left\{\!\!\begin{array}{l}
      (6-c)(6-d)-4=0,\\
      (1-c)(1-d)-4=0,
    \end{array}\right.\ \text{即}\ 
    \left\{\!\!\begin{array}{l}
      c+d=7,\\
      cd=10,
    \end{array}\right.\]
  所以 $c=2$, $d=5$ 或 $c=5$, $d=2$.
\end{solution}

\subsection{二次不等式恒成立}

\begin{example}
  如果 $kx^2-2x+6k<0$ ($k\neq 0$) 的解集为全体实数, 求 $k$ 的取值范围.
\end{example}
\begin{solution}
  题意表明对应的抛物线恒在 $x$~轴下方, 所以
  \[\left\{\!\!\begin{array}{l}
      k<0,\\
      \Delta= (-2)^2- 4k\cdot 6k<0,
    \end{array}\right.\ \text{解得}\quad
    k<-\frac{\sqrt6}6,\]
  即 $k\in\biggl(-\infty, -\dfrac{\sqrt6}6\biggr)$.
\end{solution}

关于 $x$ 的形如 $Ax^2+Bx+C>0$ 的不等式恒成立问题, 解题步骤如下:

(1) ({\bfseries 重要步骤}) 确认不等式的次数, 即考虑二次项系数 $A$ 是否为 $0$;

(2) 若 $A=0$, 则不等式化为一次不等式 $Bx+C>0$, 它恒成立的充要条件是
\[B=0\ \text{且}\ C>0;\]

(3) 若 $A\neq0$, 则不等式为二次不等式, 它恒成立的充要条件是
\[A>0\ \text{且}\ \Delta=B^2-4AC<0.\]

(4) 综合 (2)(3) 中的取值范围 (取并集).

注意, 如果已知的不等式为 $Ax^2+Bx+C\geqslant(<,\,\leqslant) 0$, 则上述解题步骤应相应调整.

\begin{example}
  如果不等式 $(a-2)x^2+2(a-2)x-4<0$ 对 $x\in\realnum$ 恒成立, 求 $a$ 的取值范围.
\end{example}
\begin{solution}
  (1) 若 $a-2=0$ 即 $a=2$, 不等式化为 $-4<0$, 恒成立.
  
  (2) 若 $a-2\neq 0$ 即 $a\neq2$, 不等式为二次不等式, 所以
  \[\left\{\!\!\begin{array}{l}
      a-2<0,\\
      \Delta= [2(a-2)]^2- 4(a-2)(-4)<0,
    \end{array}\right.\quad\text{解得}\quad
    -2<a<2,\]
  即 $a\in(-2,2)$.
  
  综上所述, 所求 $a$ 的取值范围是 $(-2,2]$.
\end{solution}

\begin{example}\label{exa:201020-2000}
  若不等式 $\dfrac{2x^2+2kx+k}{4x^2+6x+3}<1$ 对于一切实数 $x$ 都成立, 求实数 $k$ 的取值范围.
\end{example}
\begin{solution}
  因为分母的判别式 $6^2-4\cdot4\cdot3<0$ 且二次项系数 $4>0$, 所以分母恒正, 不等式化为 $2x^2+(6-2k)x+(3-k)>0$, 则
  \[\Delta= (6-2k)^2- 4\cdot2(3-k)<0,\quad\text{解得}\quad
    1<k<3,\]
  即 $k\in(1,3)$.
\end{solution}

例~\ref{exa:201020-2000} 中分母恒正也可以通过配方确定:
\[4x^2+6x+3= 4\biggl(x^2+\frac32x+\frac34\biggr)
  = 4\biggl[\biggl(x+\frac34\biggr)^2+\frac3{16}\biggr].\]

\subsection{二次方程根的分布 (选学)}
\begin{example}
  若方程 $2(m+1)x^2+4mx+3m-2=0$ 有两个负实根, 求实数 $m$ 的取值范围.
\end{example}
\begin{solution}
  由题意, 已知的方程为二次方程, 即 $2(m+1)\neq 0$. 设方程的两根为 $x_1$, $x_2$, 则这两个根均为负数的充要条件是
  \[\left\{\!\!\begin{array}{l}
      x_1+x_2=-\dfrac{4m}{2(m+1)}<0,\\
      x_1x_2=\dfrac{3m-2}{2(m+1)}>0,
    \end{array}\right.\quad\text{解得}\quad
    m<-1\ \text{或}\ m>\frac23,\]
  所求的 $m$ 的取值范围是 $(-\infty,-1)\cup\biggl(\dfrac23,+\infty\biggr)$.
\end{solution}

设关于 $x$ 的二次方程 $Ax^2+Bx+C=0$ ($A\neq0$) 的两根为 $x_1$, $x_2$, 则有以下结论 (为什么?):

(1) 两根为正的充要条件是
  \[\left\{\!\!\begin{array}{l}
      x_1+x_2>0,\\
      x_1x_2>0,
    \end{array}\right.\quad\text{即}\quad
    \left\{\!\!\begin{array}{l}
      -\dfrac{B}A>0,\\
      \dfrac{C}A>0;
    \end{array}\right.\]

(2) 两根为负的充要条件是
  \[\left\{\!\!\begin{array}{l}
      x_1+x_2<0,\\
      x_1x_2>0,
    \end{array}\right.\quad\text{即}\quad
    \left\{\!\!\begin{array}{l}
      -\dfrac{B}A<0,\\
      \dfrac{C}A>0;
    \end{array}\right.\]
    
(3) 两根一正一负的充要条件是
  \[x_1x_2<0\quad\text{即}\quad\frac{C}A<0.\]

\endinput