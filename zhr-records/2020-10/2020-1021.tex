\section{2020 年 10 月 21 日答疑记录}

\subsection{求已知类型的函数解析式}

当函数的类型已知时, 用待定系数法来解, 即只需按题意设函数解析式, 并写出其中的系数满足的条件, 再解相应的方程组即可.

\begin{example}\label{exa:201105-2130}
    已知 $f(x)$ 是一次函数, 
    \[\left\{\!\!\begin{array}{l}
        2f(1)+3f(2)=5,\\
        2f(-1)-f(0)=-1,
      \end{array}\right.\]
    求 $f(x)$ 的解析式.
\end{example}
\begin{solution}
    因为 $f(x)$ 是一次函数, 所以可以设 $f(x)=kx+b$ ($k\neq 0$). 由题意,
    \[\left\{\!\!\begin{array}{l}
        2(k+b)+3(2k+b)=5,\\
        2(-k+b)-b=-1,
      \end{array}\right.\ \text{解得}\ 
      \left\{\!\!\begin{array}{l}
        k=\dfrac59,\\[3pt]
        b=\dfrac19,
      \end{array}\right.\]
    所以 $f(x)=\dfrac59 x+ \dfrac19$.
\end{solution}

解例~\ref{exa:201105-2130} 时应注意, 为使 $f(x)$ 为一次函数, 其解析式的一次项系数非零, 即解题过程中的 $k\neq0$. 若 $f(x)$ 为二次函数, 也应注意其解析式的二次项系数非零. 其他形式可以类推.

\begin{example}\label{exa:201105-2140}
    已知 $f(x)$ 为二次函数, $f(0)=1$, $f(x+1)=f(x)+2x$, 求 $f(x)$ 的解析式.
\end{example}
\begin{solution}
    由 $f(x)$ 为一元二次函数和 $f(0)=1$ 可设 
    \[f(x)=ax^2+bx+1\quad (a\neq 0),\]
    再代入 $f(x+1)=f(x)+2x$ 可得
    \[a(x+1)^2+b(x+1)+1=(ax^2+bx+1)+2x,\]
    即 $2ax+a+b=2x$, 所以
    \[\left\{\!\!\begin{array}{l}
        2a=2,\\
        a+b=0,
      \end{array}\right.\ \text{解得}\ 
      \left\{\!\!\begin{array}{l}
        a=1,\\
        b=-1,
      \end{array}\right.\]
    即 $f(x)= x^2-x+1$.
\end{solution}

在解例~\ref{exa:201105-2140} 时, 设出解析式之后, 因为要确定两个系数 $a$, $b$, 所以也可以根据 $f(x+1)=f(x)+2x$ 取两个特殊的 $x$ 值, 得到两个方程从而解出 $a$ 和 $b$. 由于已有 $f(0)=1$, 所以不妨分别设 $x=0$ 和 $-1$, 可得 \[f(1)=0,\ f(-1)=2,\quad \text{即}\quad a+b=0,\ a-b=2.\]
用这种方法也可以方便地求出两个系数.

\subsection{求未知类型的函数解析式}

若函数解析式的类型未知, 则一般是用整体代换来解决. 例如已知 $f(x+1)=2x$ 求 $f(x)$ 时, 应理解前一表达式 (即 $f(x+1)=2x$) 中的 $f$ 是作用在 $x+1$ 上, 而后一表达式 (即 $f(x)$) 中的 $f$ 是作用在 $x$ 上. 想要求出后者相当于是弄清楚前者中的 $f$ 是如何作用在 $x+1$ 上并得出 $2x$ 的. 此时应把 $x+1$ 看成整体, 比如设 $x+1=t$, 再想办法把 $2x$ 用 $t$ 表示出来. 因为 $x=t-1$, 所以 $2x=2(t-1)$, 表明由 $f(x+1)=2x$ 可得 $f(t)=2(t-1)$, 又可写成 $f(x)=2(x-1)$. 

由以上分析可知, 解题这类问题时, 主要步骤一般只有两步: 先把 $f$ 作用的式子看成整体, 再把得到的式子用该整体表示.

\begin{example}\label{exa:201105-2150}
    设 $f(2x+1)=x^2-2x$, 求 $f(x)$ 和 $f(1)$.
\end{example}
\begin{solution}
    对 $f(2x+1)=x^2-2x$, 令 $2x+1=t$, 则 $x=\dfrac{t-1}2$, 所以
    \begin{align*}
        f(t)&=f(2x+1)=x^2-2x= \biggl(\frac{t-1}2\biggr)^2- 2\cdot\frac{t-1}2\\
        &= \frac14 t^2- \frac32 t+ \frac54,
    \end{align*}
    即 $f(x)= \dfrac14 x^2- \dfrac32 x+ \dfrac54$. 而
    \[f(1)= \frac14- \frac32+ \frac54=0.\]
\end{solution}

例~\ref{exa:201105-2150} 中的 $f(1)$ 也可通过在 $f(2x+1)=x^2-2x$ 中令 $x=0$ 得到. 类似地, 若求 $f(3)$, 则可直接令 $x=1$.

\begin{example}\label{exa:201105-2200}
    已知 $f\biggl(x+\dfrac1x\biggr)= x^2+\dfrac1{x^2}$, 求 $f(x)$.
\end{example}
\begin{solution}
    设 $x+\dfrac1x=t$, 则 
    \[x^2+\frac1{x^2}= \biggl(x+\dfrac1x\biggr)^2-2= t^2-2,\]
    所以 $f(t)= t^2-2$, 即 $f(x)=x^2-2$.
\end{solution}

例~\ref{exa:201105-2200} 比较特殊, 解题时没有根据 $x+\dfrac1x=t$ 解出 $x$ 再代入 $x^2+\dfrac1{x^2}$, 而是直接利用恒等式做了替换. 类似的恒等式还有 (注意观察恒等式的特征)
\begin{gather*}
    x^2+\frac4{x^2}= \biggl(x+\dfrac2x\biggr)^2-4,\\
    x^4+\frac1{x^4}= \biggl(x^2+\dfrac1{x^2}\biggr)^2-2
        = \biggl(\biggl(x+\dfrac1x\biggr)^2-2\biggr)^2-2.
\end{gather*}

\begin{example}\label{exa:201105-2210}
    若 $2f(x)+f\biggl(\dfrac1x\biggr)=3x$ ($x\neq 0$) 恒成立, 求 $f(x)$ 的解析式.
\end{example}
\begin{solution}
    把已知表达式中的所有 $x$ 都换成 $\dfrac1x$, 得 $2f\biggl(\dfrac1x\biggr)+f(x)=3\dfrac1x$, 再与已知表达式联立并消去 $f\biggl(\dfrac1x\biggr)$, 解得 $f(x)= 2x-\dfrac1x$ ($x\neq0$).
\end{solution}

例~\ref{exa:201105-2210} 也是特殊题, 解这种题时, 一般是想办法凑出和已知表达式类似的式子 (通常是把 $x$ 换成其他式子), 联立并解出需要的 $f(x)$. 例如, 若已知 $2f(x)+f\biggl(-\dfrac1x\biggr)=3x$, 则可以把 $x$ 都换成 $-\dfrac1x$, 可得 
\[2f\biggl(-\dfrac1x\biggr)+f(x)= -\dfrac3x;\]
若已知 $2f(x)+f(1-x)=3x$, 则可以把 $x$ 都换成 $1-x$, 可得
\[2f(1-x)+f(x)=3(1-x).\]
以上都只进行了一次替换, 即可与已知表达式联立并解出 $f(x)$. 有的时候可能需要替换多次, 才能得到能联立的方程. 例如, 若已知 $f(x)+f\biggl(\dfrac1{1-x}\biggr)=3x$, 则可以把 $x$ 都换成 $\dfrac1{1-x}$, 再把 $x$ 都换成 $\dfrac{x-1}{x}$, 就可以得到关于 $f(x)$, $f\biggl(\dfrac1{1-x}\biggr)$ 和 $f\biggl(\dfrac{x-1}{x}\biggr)$ 的三元一次方程组 (具体过程可自行写出).