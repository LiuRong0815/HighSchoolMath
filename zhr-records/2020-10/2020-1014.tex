\section{2020 年 10 月 14 日答疑记录}

本次答疑主要讲解均值不等式及其应用, 2020 年 9 月 29 日答疑记录中已有该不等式的详细推导过程, 这里仅罗列相关结论:
\begin{align*}
    \text{均值不等式:}\ &\frac{x+y}2\geqslant \sqrt{xy},\quad x,y\geqslant 0,\\
    \text{其等价形式:}\ &x+y\geqslant 2\sqrt{xy},\quad x,y\geqslant 0,\\
        &\frac{x^2+y^2}2\geqslant xy,\quad x,y\in\realnum.\\
        &x^2+y^2\geqslant 2xy,\quad x,y\in\realnum.
\end{align*}
式中 ``$=$'' 成立当且仅当 $x=y$. 此外, 如下结论 (均设 $x$, $y\geqslant 0$):

(1) 若 $xy=L$ 为定值, 则 $x+y$ 的最小值为 $2\sqrt{L}$, ``$=$'' 成立当且仅当 $x=y=\sqrt{L}$;

(2) 若 $x+y=M$ 为定值, 则 $xy$ 的最大值为 $\dfrac{M^2}4$, ``$=$'' 成立当且仅当 $x=y=\dfrac{M}2$.

用均值不等式解题时, 应想办法凑出两个式子, 使它们的积或和为定值, 如对 $x>-1$, 
\[x+\dfrac1{1+x}=(1+x)+\dfrac1{1+x}-1\geqslant 2\sqrt1-1=1.\]
除了应注意式中字母的范围 (是非负还是任意实数), 还应检验等号成立的条件, 如对 $x>2$, 虽然由均值不等式知 $x+\dfrac1x\geqslant 2$, 但等号成立当且仅当 $x=\dfrac1x$ 即 $x=1$, 与 $x>2$ 不符, 所以只能写 $x+\dfrac1x>2$.

\begin{example}
    若 $a$, $b\in\realnum$ 且 $ab>0$, 则下列不等式恒成立的是 (\qquad).
    \begin{twocolpro}
    A. $a^2+b^2>2ab$ & B. $a+b\geqslant 2\sqrt{ab}$\\
    C. $\dfrac1a+\dfrac1b> \dfrac2{\sqrt{ab}}$
    & D. $\dfrac{b}a+\dfrac{b}a\geqslant 2$
    \end{twocolpro}
\end{example}
\begin{solution}
    若只对这四个选项简单地套用均值不等式, 可能会错误地认为全都正确. 但如果仔细检查不等式成立的前提, 可发现其中有三个选项是错的.
    
    对 A, 当 $a=b$ 时, $a^2+b^2=2ab$, 所以 A 不正确.
    
    对 B, 只有 $a$, $b\geqslant 0$ 时, 原不等式才正确, 而题中的 $ab>0$ 可能有 $a$, $b<0$, 无法用均值不等式.
    
    对 C, 当 $a=b>0$ 时, $\dfrac1a+\dfrac1b= \dfrac2{\sqrt{ab}}$, 所以 C 不正确. 此外, 题中的 $ab>0$ 可能有 $a$, $b<0$, 无法用均值不等式.
    
    对 D, 由 $ab>0$ 知 $\dfrac{b}a$, $\dfrac{a}b>0$, 所以可用均值不等式得出结论.
\end{solution}

\begin{example}
    条件 ``$x>0$'' 是 ``$x^2+\dfrac1{x^2}\geqslant 2$'' 的什么条件?
\end{example}
\begin{solution}
    因为 $x^2$ 恒非负, 所以只要 $x^2\neq0$, 就有 $x^2>0$, 此时 $x^2+\dfrac1{x^2}\geqslant 2$. 这表明 $x^2+\dfrac1{x^2}\geqslant 2$ 的充要条件是 $x\neq 0$. 再与 $x>0$ 对比可知, ``$x>0$'' 是 ``$x^2+\dfrac1{x^2}\geqslant 2$'' 的充分不必要条件.
\end{solution}

\begin{example}
    若函数 $f(x)= x+\dfrac1{x-2}$ ($x>2$) 在 $x=a$ 处取最小值, 求 $a$ 的值.
\end{example}
\begin{solution}
    由 $x>2$ 知 $x-2>0$, 所以
    \[x+\dfrac1{x-2}= (x-2)+\dfrac1{x-2}+2\geqslant 2\sqrt1+2=4,\]
    等号成立当且仅当 $x-2=\dfrac1{x-2}$ 即 $x=3$ (注意 $x>2$). 故所求的 $a=3$.
\end{solution}

\begin{example}
    若 $x$, $y\in\realnum^+$ 且 $\dfrac{x}3+\dfrac{y}4=1$, 求 $xy$ 的最大值.
\end{example}
\begin{solution}
    由均值不等式, 
    \[\frac{x}3+\frac{y}4\geqslant 2\sqrt{\frac{xy}{12}},
    \quad\text{即}\quad 1\geqslant 2\sqrt{\frac{xy}{12}},\]
    解得 $xy\leqslant 3$, 所以 $xy$ 的最大值为 $3$.
\end{solution}