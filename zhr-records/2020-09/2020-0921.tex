\section{2020 年 9 月 21 日答疑记录}

\begin{example}
  集合 $A=\{x\mid x^2=1\}$, $B=\{x\mid ax=1\}$, 若 $B\subseteq A$, 求实数 $a$ 的值.
\end{example}
\begin{solution}
  集合 $A$ 描述的是方程 $x^2=1$ 的根, 即 $A=\{-1,1\}$; 集合 $B$ 描述的是方程 $ax=1$ 的根.
  
  (1) 若 $B=\varnothing$, 即方程 $ax=1$ 无根, 此时 $a=0$.
  
  (2) 若 $B\neq\varnothing$, 即方程 $ax=1$ 有根, 由该方程为一次方程知 $B= \{-1\}$ 或 $\{1\}$, 相应的 $a=-1$ 或 $1$.
  
  综上所述, $a=-1$, $0$ 或 $1$.
\end{solution}
\begin{remark}
  (1) 集合的包含关系隐含 ``小的集合可能是空集'' (如上题中的 $B$ 可能为空集, 需分类讨论).
  
  (2) 关于 $x$ 的一次方程 $ax=b$ 无解的充要条件是 $a=0$ 且 $b\neq 0$; 有解的充要条件是 $a\neq 0$ (为什么?). 
\end{remark}

\begin{example}
  (1) ``$x\in A$'' 是 ``$x\in A\cup B$'' 的 \underline{\hskip 2cm} 条件;
  
  (2) ``$x$, $y$ 为无理数'' 是 ``$x+y$ 为无理数'' 的 \underline{\hskip 2cm} 条件.
\end{example}
\begin{solution}
  (1) 因为 $A\cup B \subseteq A$, 所以 ``$x\in A$'' 是 ``$x\in A\cup B$'' 的必要不充分条件.
  
  (2) 若 $x=\sqrt2$, $y=-\sqrt2$, 则 $x+y=0$ 为有理数; 若 $x+y=\sqrt2$ 为无理数, 则可能 $x=\sqrt2$ 为无理数, $y=0$ 非无理数. 由这两个反例知, ``$x$, $y$ 为无理数'' 与 ``$x+y$ 为无理数'' 没有必然的因果关系, 所以``$x$, $y$ 为无理数'' 是 ``$x+y$ 为无理数'' 的既不充分也不必要条件.
\end{solution}
\begin{remark}
  (1) 判断两个条件的充分必要性, 一般是看两个条件对应集合的包含关系: 若 $A\subseteq B$, 则 ``$x\in A$'' 是 ``$x\in B$'' 的充分不必要条件 (简记为 ``小范围为充分的, 大范围为必要的''). 
  
  (2) 判断两个条件没有充分或必要关系, 也是应该考虑对应集合没有包含关系, 一般是举反例说明. 再举一例. 考虑 ``$x+y\geqslant 1$'' 与 ``$x^2+y^2\geqslant 1$'' 的关系. 若 $x+y\geqslant 1$, 不妨取 $x=y=\dfrac12$, 则 $x^2+y^2=\dfrac12< 1$; 若 $x^2+y^2\geqslant 1$, 不妨取 $x=0$, $y=1$, 则 $x+y=-1< 1$. 由这两个反例知, ``$x+y\geqslant 1$'' 是 ``$x^2+y^2\geqslant 1$'' 的既不充分也不必要条件.
\end{remark}

% \end{document}
