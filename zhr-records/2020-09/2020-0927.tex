\section{2020 年 9 月 27 日答疑记录}

\begin{example}\label{exa:2020-1012-1900}
  解关于 $x$ 的不等式 $x^2-(2+a)x+2a<0$.
\end{example}
\begin{solution}
  不等式化为 $(x-2)(x-a)<0$, 所以根据二次函数 $y=(x-2)(x-a)$ 的图象知,
  
  若 $a<2$, 则 $x\in (a,2)$; 
  
  若 $a=2$, 则 $x\in \varnothing$; 
  
  若 $a>2$, 则 $x\in (2,a)$.
\end{solution}

例~\ref{exa:2020-1012-1900} 中关于 $x$ 的不等式虽然系数中带了参数 $a$, 但是仍然可以用因式分解的方法求得其解集. 由于系数中带了参数, 所以对应的二次函数图象由参数决定, 写解集时需要分类讨论, 讨论的主要依据是图象与 $x$ 轴交点的坐标. 再看一个复杂一点的例子.

\begin{example}\label{exa:2020-1012-1910} 
  解关于 $x$ 的不等式:
  \begin{twocolpro}
    (1) $x^2+2x+ax+2a>0$; & (2) $2x^2+(2+a)x+a\geqslant 0$;\\ 
    (3) $x^2+ax-6a^2\leqslant 0$. &
  \end{twocolpro}
\end{example}
\begin{solution}
  (1) 不等式化为 
  \[x^2+(2+a)x+2a>0\quad\text{即}\quad (x+2)(x+a)>0,\]
  所以根据二次函数 $y=(x+2)(x+a)$ 的图象知, 
  
  若 $-a<-2$ 即 $a>2$, 则 $x\in (-\infty,-a)\cup (-2,+\infty)$; 
  
  若 $-a=-2$ 即 $a=2$, 则 $x\in \{x\mid x\neq-2\}$; 
  
  若 $-a>-2$ 即 $a<2$, 则 $x\in (-\infty,-2)\cup (-a,+\infty)$.
  
  (2) 不等式化为 $(2x+a)(x+1)\geqslant 0$, 所以根据二次函数 $y=(2x+a)(x+1)$ 的图象知, 
  
  若 $-\dfrac{a}2<-1$ 即 $a>2$, 则 $x\in \biggl(-\infty,-\dfrac{a}2\biggr]\cup [-1,+\infty)$; 
  
  若 $-\dfrac{a}2=-1$ 即 $a=2$, 则 $x\in \realnum$; 
  
  若 $-\dfrac{a}2>-1$ 即 $a<2$, 则 $x\in (-\infty,-1]\cup \biggl[-\dfrac{a}2,+\infty\biggr)$.
  
  (3) 不等式化为 $(x+3a)(x-2a)\leqslant 0$, 所以根据二次函数 $y=(x+3a)(x-2a)$ 的图象知, 
  
  若 $-3a<2a$ 即 $a>0$, 则 $x\in [-3a,2a]$; 
  
  若 $-3a=2a$ 即 $a=0$, 则 $x\in \{0\}$; 
  
  若 $-3a>2a$ 即 $a<0$, 则 $x\in [2a,-3a]$.
\end{solution}

从例~\ref{exa:2020-1012-1910} 可以看出, 解系数带参数的关于 $x$ 的不等式, 步骤一般为: 把不等式整理为 $Ax^2+Bx+C>0$ 的形式 (建议 $A>0$), 再对二次式因式分解, 接着讨论对应二次方程的根的大小, 最后根据讨论的情况和二次函数图象写出对应的解集. 


